%%% A template for a simple PDF/A file like a stand-alone abstract of the thesis.

\documentclass[12pt]{report}

\usepackage[a4paper, hmargin=1in, vmargin=1in]{geometry}
\usepackage[a-2u]{pdfx}
\usepackage[utf8]{inputenc}
\usepackage[T1]{fontenc}
\usepackage{lmodern}
\usepackage{textcomp}

\begin{document}
Tradiční hudební doporučovací systémy využívají metody kolaborativního filtrování. To je ovšem nevýhoda pro posluchače, kteří preferují méně mainstreamové skladby, protože kolaborativní filtrování je závislé na popularitě skladeb. Doporučování na základě obsahu by mohlo být rozumná volba při řešení tohoto problému. Vzhledem k tomu, že vyhledávání na základě tagů je rozšířené při napomáhání tradičním hudebním doporučovacím systémum, v této práci představujeme jiné "content-based" metody, které stanovují podobnost skladeb na základě využití textu a hudby. 
Jako první jsme vyhodnotili správnost doporučování několika textových a hudebních metod na playlistech skutečných uživatelů a zjistili, že textové metody mají výsledky konkurence schopné v porovnání s audio metodami. Výsledky také odhalily, že v obou kategoriích jsou metody, které jsou 100 krát lepší než náhodné dopourčování a mají potenciál ke zlepšení.
Po vyhodnocovací fázi jsme vybrali kvalitní metody a implementovali je do webové aplikace, která má za cíl doporučovat novou hudbu uživatelům podle dle preferencí.

\end{document}
