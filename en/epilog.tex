\chapter*{Conclusion}
We tested multiple content-based recommendation techniques and the results show that both lyrics-based and audio-based methods are able to provide relevant recommendations. The numbers suggest that for some methods the recommendations are more than hundred times better than random recommendation. 

A closer examination of the results also indicates that lyrics-based methods -- specifically the Tf-idf and PCA with Tf-idf as input --- profit from the variability of items they recommend which appears to suit real user playlists well. The deep audio-based methods, even though they stayed a bit behind in performance, seem to be prospective methods especially with additional hyper-parameter and architecture adjustments. 

We also successfully introduced a running web application with recommendation methods that provide users with novel, relevant suggestions without being dependant on song popularity. The fact that our dataset, which was loaded into the application, probably contains mostly at least somewhat popular songs is compensated by the fact that each user can add his own songs. Like this less popular songs can become part of the database and have an equal change to be recommended as popular songs. 

\section*{Future work}
\subsection*{Recommendation methods}
Further study of lyrics-based methods is one way to go in future work. We could use for example the Tf-idf vectors not only as input for the PCA or SOM but also for other types of neural networks. They might not be suited for RNN networks as there is no kind of sequential information stored in the Tf-idf vector but it would be interesting to try different architectures.

As we tested our audio methods, some proved to be more perspective than others. Mel-spectrograms and MFCCs seem to be a better input for similarity methods than raw spectrograms. Also the PCA showed to have great potential with both lyrics and audio based methods so further research into this and other dimensionality reduction methods seems to be a good idea. 

When it comes to neural networks, networks with the "GRU" layers seem to perform better than "LSTM" layers. Also many different layer combinations and architectures can be tested. Using RNNs without returning whole sequences also seems as a reasonable thing to do because it appears that the fact that the sequence-to-sequence RNN autoencoders only reduce the number of features and not the number of timestamps is limiting for these networks. The PCA which reduced the features and the timestamps might have benefited mainly from that. Also other types than RNNs can be used to build the autoencoders which then can be tested and possibly implemented.

Moreover, besides making changes in the methods that encode songs into vectors, research could be also done on the aggregation of vectors by some similarity measure. We used only cosine similarity throughout the thesis but apart from other simple distance metrics such as the \textit{Euclidean distance} or \textit{Manhattan distance} more advanced aggregation methods can be tested for example the GRU4rec \cite{gru4rec_article} method might be a suitable metric especially if we had data about songs that were played most recently and would like to introduce session-based recommendation.

\subsection*{Web application}
The web application can be further extended in multiple ways. One thing would be creating a system, where the users could rate the recommendations so we would have feedback about method performance not only from the evaluation we did in this thesis but also from real time users. 

Also, more advanced recommendation metrics could be applied in the web application. We could keep track on the users lastly played songs or take into account how many times he played a song and then use this in the final similarity calculation. For example have something like \textit{The most similar songs to the last 10 songs you played} or \textit{The most similar songs to your 10 most played songs} etc. 

There is obviously also the possibility of including more similarity methods into the application, which however now involves a non-trivial amount of changes to the source code. A simplification and better design pattern for the logic of the application could be a step to take in the future. 

Then there are some application features that could be implemented which follow the design of traditional music-applications. This includes playing whole playlists or creating an endless playlist from the recommendations as well as adding videos and tags to songs, allowing searching based on genres etc. 


\addcontentsline{toc}{chapter}{Conclusion}
