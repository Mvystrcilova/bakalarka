\chapter*{Conclusion}
The goal of this this thesis was to test and implement multiple content-based recommendation techniques. On the way, we dealt with many various tasks from building a web application (which has it perks itself for example dealing with the database, python package compatibility, server configuration, celery configuration, ...) to extracting music audios, handling GPU drivers, building neural networks etc. At the end, we can say that we successfully introduced a running web application with recommendation methods that provide users with relevant recommendations (compared to for example random recommendations). \\
Our most successful method -- the PCA\_Tf-idf method -- is lyrics based which was not expected. From that we can conclude that creating lyrics-based recommendation applications is not unreasonable. Audio methods also proved to yield good results.

\section*{Future work}
\subsection*{Recommendation methods}
Further study of lyric-based method is one way to go in future work. We could use for example the Tf-idf vectors not only as input for the PCA or SOM but also for other types of neural networks. They might not be suited for RNN networks as there is no kind of sequential information stored in the Tf-idf vector but it would be interesting to try different architectures. \\
As we tested our audio methods, some proved to be more perspective than others. Mel-spectrograms and MFCCs seem to be a better input for similarity methods than raw spectrograms. Also the PCA showed to have great potential with both lyrics and audio based methods so further research into this seems to be a good idea. \\
When it comes to neural networks, networks with the "GRU" layers seem to perform better than "LSTM" layers. Also various different architectures can be tested. Using other types of neural networks than RNNs seems as a reasonable thing to do because it appears that reducing only the number of features is limiting for these networks, as the PCA which reduced the features and the samples might have benefited mainly from that. \\
\subsection*{Web application}
The web application can be further extended in multiple ways. One thing would be creating a system, where the users could rate the recommendations so we would have feedback about method performance not only from the evaluation we did in this thesis but also from real time users. \\
Also more advanced recommendation metrics could be applied in the web application. We could keep track on the users lastly played songs or take into account how many times he played a song and then use this in the final similarity calculation. For example have something like \textit{The most similar songs to the last 10 you played} or \textit{The most similar songs to your 10 most played songs} etc. \\
There is obviously also the possibility of including more similarity methods into the application, which however now involves a non-trivial amount of changes to the source code. So maybe also a simplification and better design pattern for the "logic" of the application could be a step to take in the future. \\
Then there are some application features that could be implemented which follow the design of traditional music-applications. This includes playing whole playlists or creating an endless playlist from the recommendations as well as adding videos and tags to songs and allow searching based on genres etc. \\

\addcontentsline{toc}{chapter}{Conclusion}
