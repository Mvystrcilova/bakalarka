\chapter{Web Application}

In this section, we will describe the pro-posed web application for novel song recommendation.  The section is structured as follows: 
\begin{itemize}
    \item 1. We analyze our application goals and describe what the user can expect from our application.
    \item 2. We describe the methods we used in our application and the obstacles we had to overcome with their individual implementations to provide the user with the behaviour as described in 1.
    \item 3. We briefly introduce the building blocks of our application with focus on the calculation of recommendations.
    \item 4. We present the possible configurations of our application.
\end{itemize}

\section{Analysis}

There are many music recommendation web applications online such as Youtube, Spotify etc. They have a lot of data about users, user activity, a lot of songs, a lot of tags. Our application is just a small project that does not aspire on growing to such extend. What we want to provide to our users are not endless playlists but more of an inspiration to find new songs and then play them (for exapmle on Youtube). \\
We obviously want our application to have the usual web application functionalities, such as creating and account, logging in and out, going through different web pages, etc. Because it is a song recommendation application, we want our users to be able to view their recommendations, create playlists and search for songs and like and dislike songs to improve the recommendations. Besides that we want the users to be able to add songs they already know and that are missing in our database and then see, which similarity methods yield which recommendations. Even if he does not like the recommended song, it might be interesting for him to see, that some song was recommended to him based on another song when looking at for example lyrics. It gives the user a more hands on experience and is so not only about music but also a bit about statistics and math. I know, this might limit the user pool to nerds, but oh well.\\
To do this, we need to 