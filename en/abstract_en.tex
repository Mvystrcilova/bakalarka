%%% A template for a simple PDF/A file like a stand-alone abstract of the thesis.

\documentclass[12pt]{report}

\usepackage[a4paper, hmargin=1in, vmargin=1in]{geometry}
\usepackage[a-2u]{pdfx}
\usepackage[utf8]{inputenc}
\usepackage[T1]{fontenc}
\usepackage{lmodern}
\usepackage{textcomp}

\begin{document}
Traditional music recommender systems rely on collaborative-filtering methods. This, however, puts listeners who do not enjoy mainstream songs at a disadvantage because CF systems depend on popularity patterns. Content-based recommendation methods might be useful in solving this issue. Since tag-based searches are a widespread tool to aid traditional music recommendation, this paper presents content-based methods measuring similarity between songs with focus on methods utilizing song's lyrics and audio recordings.
First, we evaluated the accuracy of several approaches based on lyrics and audio information on real user playlists and found that lyrics-based methods yield competitive results to audio-based methods. Results also revealed that both categories include methods that are 100 times more accurate compared to random suggestions and that they have potential for even better results. After the evaluation phase, we selected well-performing methods and implemented them in a web application aiming on recommending novel music to the users based on their content-based profile.
\end{document}