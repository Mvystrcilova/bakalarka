\chapter{Introduction}

Millions of songs online provide an opportunity to find great songs for people with all kinds of music tastes. However, only a small fraction of the magnitude of all the songs produced becomes popular and those are the ones people are being exposed to the most. \todo{LP: They, them - neprehanejte to s temi zajmeny} They are recommended to them on various platforms such as YouTube or Spotify, and played across all radio stations sometimes multiple\todo{LP: several times a day} times a day. After a while, older songs become less popular, one could use the term \textit{"overplayed"} and other (usually new) songs take their place. But what if a person's next favorite song already existed, it just did not become popular? It is unlikely to hear unpopular but possibly likeable songs for a person with unusual music preferences on the radio. Radio stations try to target as many listeners as possible.However with the help of a recommender collecting data about what a user listens to, it could help anyone could discover tracks perfectly tailored for them without being dependant on their popularity. \\
The help of recommendation systems for basically any online content is crucial. With the amount of songs as mentioned earlier but also movies, books, clothes, electronics and many more, it would be extremely time consuming for a person to go through all of it in order to find what they are looking for. Recommendation systems are trying to make it easier for people to find what they want for and even predict, what they will be looking for next or what they might want but maybe just do not know it yet. \\
There are three main method groups to generate (not only) music recommendations for users. First are collaborative-filtering methods (CF) where recommendations are based on like-minded users preferences, second are content-based methods where recommendations are based on the song content (tags, audio, lyrics, ...) and the third group are hybrid methods combining the first two together. \\
Generally, CF methods for not just music but all kinds of recommendation systems appear to be researched more extensively [5] however, there are certain drawbacks these approaches. Most obviously, there is a problem with new, unrated songs because no user has viewed or liked them, so they cannot be recommended to like-minded users with a method based only on collaborative filtering. This is called the cold-start problem. Also, the recommendations tend to be dependent on user popularity patterns. Nevertheless, with enough user data, collaborative filtering methods generally outperform content-based methods \cite{van2013deep}. \\
Due to this, there are not many applications that would recommend songs based solely on their content and to the best of our knowledge there is no music recommendation application that would recommend songs to its users based only on lyrics. As this is a logical consequence of the findings above, we believe that a recommendation system based exclusively on content-based methods could be helpful for users with an unusual taste. Another reason to study these methods is the lack of research for lyrics-based methods. There is an intuitive a notion that their performance might be doubtful, however, many studies evaluate them based on their genre classification accuracy \cite{DBLP:journals/corr/Tsaptsinos17} or compare them to collaborative filtering systems \cite{Gossi2016LyricBasedMR}. It also makes sense to focus on content-based methods as there will be no users in my web application at first which would make it a really rough cold start. \\
Because of that the goal of this thesis is to describe various unsupervised text and audio based machine learning methods for content-based song similarity calculations, choose some of them based on their features and previous studies, evaluate their performance on real user playlists, compare their performance and then implement chosen ones in a web application.\\
During our research it turned out that measuring song similarity based on lyrics and based on audio are two completely different tasks and there are multiple sub-problems in both approaches such as language, feature representation and similarity metrics for lyrics based methods and audio extraction, audio representation, feature extraction and similarity metrics for audio based similarity. As it would be possible to write a thesis about each of these sub-tasks we decided to focus on unsupervised learning of song feature representation. That means encoding a song into a vector so that a simple distance metric can be used to evaluate similarity of two arbitrary songs without having any information about genre or other tags. The methods will be divided into two main groups. In the first one, songs will be encoded based on their lyrics in the second one based on audio. In both groups, we described basic as well as more advanced existing unsupervised learning methods. We compare the text and audio recommendations based on these encoding methods but also compare methods within these two groups. \\
The web applications main purpose is to introduce the user to new songs he has not listened to yet based on a feature extraction method he chooses. The songs are provided by the applications default database but adding songs is possible too and their distance to other songs is taken into account for recommendations. \\


