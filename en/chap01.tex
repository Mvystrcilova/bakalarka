\chapter{Introduction}

Millions of songs online provide an opportunity to find great songs for people with all kinds of music tastes. However, only a small fraction of the magnitude of all the songs produced becomes popular and those are the ones people are being exposed to the most. They are recommended to them on various platforms such as YouTube \footnote{https://www.youtube.com} or Spotify \footnote{https://www.spotify.com}, and played across all radio stations sometimes several times a day. After a while, older songs become less popular, one could use the term \textit{"overplayed"} and other (usually new) songs take their place. But what if a person's next favorite song already existed, it just did not become popular? It is unlikely to hear unpopular but possibly likeable songs for a person with unusual music preferences on the radio. Radio stations try to target as many listeners as possible. However with the assistance of a recommender collecting data about what a user listens to, it could help anyone discover tracks perfectly tailored for them without being dependant on their popularity.\\
The suggestions of recommendation systems for basically any online content is crucial. With the amount of songs as mentioned earlier but also movies, books, clothes, electronics and many more, it would be extremely time consuming for a person to go through all of it in order to find what they are looking for. Recommendation systems are trying to make it easier for people to find what they want for and even predict, what they will be looking for next or what they might want but maybe just do not know it yet. \\
There are three main method groups to generate (not only) music recommendations for users. First are collaborative-filtering methods (CF) where recommendations are based on like-minded users preferences, second are content-based methods where recommendations are based on the song content (tags, audio, lyrics, ...) and the third group are hybrid methods combining the first two together. \\
Generally, CF methods for all kinds of recommendation systems appear to be researched more extensively \cite{DBLP:journals/corr/abs-1712-07525} however, there are certain drawbacks these approaches. Most obviously, there is a problem with new, unrated songs because no user has viewed or liked them, so they cannot be recommended to like-minded users with a method based only on collaborative filtering. This is called the cold-start problem. Also, the recommendations tend to be dependent on user popularity patterns. Nevertheless, with enough user data, collaborative filtering methods generally outperform content-based methods \cite{van2013deep}. \\
Due to this, there are not many applications that would recommend songs based solely on their content and to the best of our knowledge there is no music recommendation application that would recommend songs to its users based only on lyrics. As this is a logical consequence of the findings above, we believe that a recommendation system based exclusively on content-based methods could be helpful for users with an unusual taste because collaborative filtering is popularity-dependant. Therefore the goal of this thesis to introduce such an application. \\
To create a content-based music recommender application we need to decide based on what content it will be recommending. A basic CB representation of songs is attribute-based. Attributes are the song's genre, the artist, creation year and so on. We are not using it in our application however because almost all music related application allow users to search based on tags. \\
We chose lyrics and audio as the content songs have to base the recommendations of the application on. Audio is something that every song has and it also is probably what is most important about it. People listen to music because it is a pleasant sound and it is likely that it is audio features that define, whether a person likes a song or not. \\
With lyrics it is a bit different. The main reason to study these methods is the lack of research for lyrics-based methods. There is an intuitive a notion that their performance might be doubtful, however, many studies evaluate them on their genre classification accuracy \cite{DBLP:journals/corr/Tsaptsinos17} or compare them to collaborative filtering systems \cite{Gossi2016LyricBasedMR} which does not always mimic actual user behaviour. \\
Recommendation is mostly based on similarity between items which can be defined in various ways. Both lyrics and audio need pre-processing before establishing similarity.  Because of that the goal of this thesis is to describe and test various ways of pre-processing text and audio in an unsupervised manner for content-based song similarity calculations, choose some of them based on previous studies and features, evaluate their performance on real user playlists, compare their performance and then implement chosen ones in a web application.\\

During our research it turned out that measuring song similarity based on lyrics and based on audio are two completely different tasks and there are multiple sub-problems in both approaches. Language, feature representation and similarity metrics for lyrics based methods and audio extraction, audio representation, feature extraction and similarity metrics for audio based similarity. We decided to focus on unsupervised learning of song feature representation which means encoding a song into a vector so that a standard similarity-based recommendation technique can be used to evaluate similarity of two arbitrary songs without having any information about genre or other tags. The vectors can also be used for more advanced algorithms using for example Recurrent neural networks to calculate similarity. That is however above the scope of this work. \\
The methods will be divided into two main groups. In the first one, songs will be encoded based on their lyrics in the second one based on audio. In both groups, we described basic as well as more advanced existing unsupervised learning encoding methods. We compare the text and audio recommendations based on these encoding methods but also compare methods within these two groups. \\
The web applications main purpose is to introduce the user to new songs he has not listened to yet based on a feature extraction method he chooses. The songs are provided by the applications default database but adding songs is possible too and their distance to other songs is taken into account for recommendations. \\


